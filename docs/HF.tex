% Created 2025-07-23 Wed 00:27
% Intended LaTeX compiler: pdflatex
\documentclass[11pt]{article}
\usepackage[utf8]{inputenc}
\usepackage[T1]{fontenc}
\usepackage{graphicx}
\usepackage{longtable}
\usepackage{wrapfig}
\usepackage{rotating}
\usepackage[normalem]{ulem}
\usepackage{amsmath}
\usepackage{amssymb}
\usepackage{capt-of}
\usepackage{hyperref}
\usepackage{physics}
\usepackage{bm}
\usepackage{geometry}
\geometry{a4paper, margin=1in}
\usepackage{hyperref}
\usepackage{listings}
\usepackage{mathrsfs}
\author{Rafael Corella}
\date{\today}
\title{Hartree-Fock theory}
\hypersetup{
 pdfauthor={Rafael Corella},
 pdftitle={Hartree-Fock theory},
 pdfkeywords={},
 pdfsubject={},
 pdfcreator={},
 pdflang={English}}
\begin{document}

\maketitle
\tableofcontents

\section{SCF Implementation}
\label{sec:orga4be61f}

\subsection{Roothaan equations}
\label{sec:org28d081b}

\begin{itemize}
\item Once the spin is integrated out, the HF equations in spatial orbital form are

\[ f(r_1)\psi_i(r_1) = \epsilon_i\psi_i(r_1) \]

\item Introducing a basis

\[ \psi_i = \sum_{\mu = 1}^K C_{\mu i }\phi_{\mu} \]

leads to the integrated HF equations in matrix form, the \textbf{Roothaan} equations

\[ FC = SC\epsilon \]

\begin{itemize}
\item The fock matrix \(F\) has elements

\[ F_{\mu\nu} = \int \dd{r_1}\phi_{\mu}^{* }(1)f(1) \phi_{\nu}(1) \]

\item The overlap matrix \(S\) has elements

\[ S_{\mu\nu} = \int \dd{r_1} \phi_{\mu}^{ * }(1) \phi_{\nu}(1) \]

\item The expansion coefficients \(C_{\mu i}\) form a \(K\times K\) matrix \(C\)

\[ C = \pmqty{C_{11} & C_{12} & \cdots & C_{1K} \\ C_{21} & C_{22} & \cdots & C_{2K} \\ \vdots & \vdots & & \vdots \\ C_{K1} & C_{K2} & \cdots & C_{KK} } \]

Las columnas de esta matriz describen a los orbitales moleculares, i.e. \(\psi_i = \sum_{\mu}C_{\mu i}\phi_{\mu}\)

\item The orbital energies \(\epsilon_i\) in matrix form are

\[ \epsilon = \pmqty{\epsilon_1 & & & \\ & \epsilon_2 & & 0 \\ 0 & & \ddots  \\ &&& \epsilon_k} \]
\end{itemize}
\end{itemize}
\subsection{Density matrix}
\label{sec:orgc39d8fc}

In restricted HF (for closed-shell) molecules, the charge density is

\[ \rho(r) = 2 \sum_a^{N/2} |\psi_a(r)|^2 \]

Introducing the basis expansion gives

\begin{align*}
    \rho(r) &= \sum_{\mu\nu}\qty[2\sum_a^{N/2}C_{\mu a}C_{\nu a}^{ * }]\phi_{\mu}(r)\phi_{\nu}^{ * }(r) \\
    \rho(r) &= \sum_{\mu\nu} P_{\mu\nu}\phi_{\mu}(r)\phi_{\nu}^{ * }(r)
\end{align*}

then the density matrix is

\begin{equation}
\label{eq:P-mat}
P_{\mu\nu}= 2 \sum_a^{N/2} C_{\mu a}C_{\nu a}^{ * }
\end{equation}
\subsection{Fock matrix}
\label{sec:org20beb46}

The matrix representation of the Fock operator \(f(1) = h(1) + \displaystyle\sum_a^{N/2} 2J_a(1) - K_a(1)\) in the basis \(\{\phi_{\mu}\}\) is

\begin{align*}
    F_{\mu\nu} &= \int\dd{r_1} \phi_{\mu}^{ * }(1)h(1)\phi_{\mu}(1) + \sum_a^{N/2}\int\dd{r_1}\phi_{\mu}^{ * }(1)[2J_a(1)-K_a(1)]\phi_{\nu}(1) \\
    F_{\mu\nu} &= H_{\mu\nu}^{\text{core}} + \sum_a^{N/2} 2 (\mu\nu|aa) - (\mu a|a \nu)
\end{align*}

\begin{itemize}
\item Here, the core-Hamiltonian matrix is

\[ H_{\mu\nu}^{\text{core}} = \int \dd{r_1} \phi_{\mu}^{ * }(1)h(1)\phi_{\mu}(1) \]

to evaluate this matrix, we need the kinetic energy integrals and the nuclear attraction integrals

\[ H_{\mu\nu}^{\text{core}} = T_{\mu\nu}+ V_{\mu\nu}^{\text{nucl}} \]

This matrix need only be evaluated once in the SCF procedure
\end{itemize}


\begin{itemize}
\item In the Fock matrix, when we insert the linear expansion of the molecular orbitals into the two-electron terms
\end{itemize}


\begin{align*}
    F_{\mu\nu} &= H_{\mu\nu}^{\text{core}} + \sum_a^{N/2}\sum_{\lambda\sigma}C_{\lambda a}C_{\sigma a}^{ * }[2(\mu\nu|\sigma\lambda) - (\mu\lambda|\sigma\nu)] \\
    F_{\mu\nu} &= H_{\mu\nu}^{\text{core}} \sum_{\lambda\alpha} P_{\lambda\sigma}[(\mu\nu|\sigma\lambda) - \frac{1}{2}(\mu\lambda|\sigma\nu)] \\
    F_{\mu\nu} &= H_{\mu\nu}^{\text{core}} + G_{\mu\nu}
\end{align*}

where \(G_{\mu\nu}\) is the two-electron part of the Fock matrix

\begin{equation}
\label{eq:G-mat}
G_{\mu\nu}= \sum_{\lambda\alpha} P_{\lambda\sigma}[(\mu\nu|\sigma\lambda) - \frac{1}{2}(\mu\lambda|\sigma\nu)]
\end{equation}

\begin{itemize}
\item The two-electron integrals are

\[ (\mu\nu|\lambda\sigma) = \int\dd{r_1}\dd{r_2} \phi_{\mu}^{ * }(1)\phi_{\nu}(1)r_{12}^{-1}\phi_{\lambda}^{ * }(2)\phi_{\sigma}(2) \]
\end{itemize}
\subsection{Canonical orthogonalization}
\label{sec:org8df9bea}

The condition that a transformation matrix \(X\) must obey in order to form an orthonormal set

\[ \int\dd{r} \phi_{\mu}'^{ * } (r) \phi_{\nu}' (r) = \delta_{\mu\nu}\]

is

\[ X^{\dagger}S X = \mathbb{I} \]

Canonical orthogonalization uses a transformation matrix

\begin{equation}
\label{eq:X-mat}
X = Us^{-1/2}
\end{equation}

where \(U\) is a unitary matrix that diagonalizes \(S\)

\[ U^{\dagger}SU=s \]

and \(s\) is a diagonal matrix of the eigenvalues of \(S\)

Consider a new coefficient matrix \(C'\) related to the old coefficient matrix \(C\) by

\begin{equation}
\label{eq:C-trans}
C' = X^{-1}C \qc C = X C'
\end{equation}

substituting \(C\) into the Roothaan equations gives

\[ F XC' = SXC' \epsilon \]

if we define a new matrix \(F'\) by

\begin{equation}
\label{eq:F-trans}
F' = X^{\dagger} FX
\end{equation}

and use the orthonormality condition on \(S\), \(X^{\dagger}SX=\mathbb{I}\), then by multiplying on the left by \(X^{\dagger}\)

\begin{equation}
\label{eq:rooth-trans}
F' C' = C' \epsilon
\end{equation}

this transformed Roothaan equations can be solved for \(C'\) by diagonalizing \(F'\), which can be used to get \(C\).

Therefore, given \(F\), we can use (\ref{eq:rooth-trans}), (\ref{eq:F-trans}), and (\ref{eq:C-trans}) to solve the Roothaan equations \(FC=SC\epsilon\) for \(C\) and \(\epsilon\)
\subsection{SCF procedure}
\label{sec:org5a6347b}

The Hartree-Fock limit is when the basis set is essentially complete and Hartree-Fock procedure is used in this case, but it can also be used in this case interchangeable as SCF procedure

\begin{enumerate}
\item Specify a molecule, which is a set of nuclear coordinate \(\{R_A\}\), atomic numbers \(\{Z_A\}\), and number of electrons \(N\), then a basis set \(\{\phi_{\mu}\}\)

\item Calculate all required molecular integrals, \(S_{zmzn}\), \(H_{\mu\nu}^{\text{core}}\), and \((\mu\nu|\lambda\sigma)\)

\item Diagonalize the overlap matrix \(S\) and obtain a transformation matrix \(X\) from (\ref{eq:X-mat})

\item Obtain a guess at the density matrix \(P\)

\item Calculate the matrix \(G\) of equation (\ref{eq:G-mat}) from the density matrix \(P\) and the two-electron integrals \((\mu\nu|\lambda\sigma)\)

\item Add \(G\) to the core-Hamiltonian to obtain the Fock matrix \(F = H^{\text{core}} + G\)

\item Calculate the transformed Fock matrix \(F' = X^{\dagger}FX\)

\item Diagonalize \(F'\) to obtain \(C'\) and \(\epsilon\)

\item Calculate \(C= X C'\)

\item Form a new density matrix \(P\) from \(C\) using (\ref{eq:P-mat})

\item Determine wether the procedure has converged
\end{enumerate}
\subsection{Integral evaluation with 1s primitive Gaussians}
\label{sec:orgf72c074}

\subsection{Obara-Saika}
\label{sec:org89261be}

\subsubsection{Two-electron integrals}
\label{sec:org5a91fd2}

Recurrence expressions for the electron repulsion integrals over s and p Cartesian Gaussian functions

\begin{itemize}
\item \((ss,ss)^{(0)} = (\zeta + \eta)^{-1/2} K(\zeta_a,\zeta_b,A,B)K(\zeta_c,\Zeta_d,C,D)F_0(T)\)
\item \((p_i s,ss)^{(0)} = (P_i - A_i)(ss,ss)^{(0)} + (W_i - P_i)(ss,ss)^{(1)}\)
\item \((p_i s, p_k s)^{(0)} = (Q_K - C_k)(p_i s,ss)^{ (0)} + (W_k - Q_k)(p_i s, ss)^{(1)} + \frac{\delta_{ik}}{2(\zeta + \eta)}(ss,ss)^1\)
\item \((p_ip_j,ss)^{(0)} = (P_j-B_j)(p_i s,ss)^{(0)} + (W_j-P_j)(p_i s,ss)^{(1)} + \frac{\delta_{ij}}{2\zeta}\{ (ss,ss)^{(0)} - \frac{\rho}{\zeta}(ss,ss)^{(1)} \}\)
\item \((p_ip_j,p_k s)^{(0)} = (Q_k - C_k)(p_ip_j,ss)^{(0)} + (W_k-Q_k)(p_ip_j,ss)^{(1)} + \frac{1}{2(\zeta + \eta)} \{ \delta_{ij}(sp_j,ss)^{(1)} + \delta_{jk}(p_i s,ss)^{(1)} \}\)
\item \((p_i p_j, p_k p_l)^{(0)} = (Q_l - D_l)(p_ip_j, p_k s)^{(0)} + (W_l - Q_l)(p_i p_j, p_k s)^{(1)} + \frac{1}{2(\zeta+\eta)}\{ \delta_{il}(sp_j,p_k s)^{(1)} + \delta_{jl}(p_i s, p_k s)^{(1)} \} + \frac{\delta_{kl}}{2\eta}\{ (p_i p_j, ss)^{(0)} - \frac{\rho}{\eta}(p_ip_j,ss)^{(1)} \}\)
\end{itemize}

The variables involved are

\begin{itemize}
\item \(\zeta = \zeta_a + \zeta_b\)
\item \(P = \frac{\zeta_a A + \zeta_b B}{\zeta}\)
\item \(G = \frac{\zeta P + \zeta_c C}{\zeta + \zeta_c}\)
\item \(\eta = \zeta_c + \zeta_d\)
\item \(\rho = \frac{\zeta \eta}{\zeta + \eta}\)
\item \(Q = \frac{1}{\eta}(\zeta_c C + \zeta_d D)\)
\item \(W = \frac{\zeta P + \eta Q}{\zeta + \eta}\)
\item \(F_m(T)\) is the mth order Boys function
\item \(T = \rho(P-Q)^2\)
\item \(K(\zeta,\zeta',R,R') = 2^{1/2} \frac{\pi^{5/4}}{\zeta + \zeta'} exp(- \frac{\zeta\zeta'}{\zeta+\zeta'}(R-R')^2)\)
\end{itemize}

Then the auxiliary repulsion integral \((ab,cd)^{(m)}\) is defined by

\begin{itemize}
\item \((ab,cd)^{(m)} = \frac{2}{\pi^{1/2}} \int_0^{\infty} du (\frac{u^2}{\rho + u^2})^m (ab|u|cd)\)

where the integral in \(u\) is introduced by the integral representation of the \(r_{12}^{-1}\) operator in \((ab,cd)\) and in particular, the true ERI is \((ab,cd)^{(0)}\)
\end{itemize}
\subsubsection{One-electron integrals}
\label{sec:org11480df}

Recurrence expressions for the nuclear attraction integrals

\begin{itemize}
\item \((s|A_0|s)^{(0)} = 2(\frac{\zeta}{\pi})^{1/2} (s||s) F_0(U)\)
\item \((p_i|A_0|s)^{(0)} = (P_i - A_i)(s|A_0|s)^{(0)} - (P_i-C_i)(s|A_0|s)^{(1)}\)
\item \((p_i|A_0|p_j)^{(0)} = (P_j - B_j)(p_i|A_0|s)^{(0)} - (P_j-C_j)(p_i|A_0|s)^{(1)}\)
\item \((d_{ij}|A_0|s)^{(0)} = (P_j-A_j)(p_i|A_0|s)^{(0)} - (P_j-C_j)(p_i|A_0|s)^{(1)}\)
\(+ \frac{\delta_{ij}}{2\zeta} \{(s|A_0|s)^{(0)}-(s|A_0|s)^{(1)}\}\)
\item \((d_{ij}|A_0|p_k)^{(0)} = (P_k-B_k)(d_{ij}|A_0|s)^{(0)} - (P_k-C_k)(d_{ij}|A_0|s)^{(1)}\)
\(+\frac{\delta_{ik}}{2\zeta} \{(p_j|A_0|s)^{(0)} - (p_j|A_0|s)^{(1)}\}\)
\(+\frac{\delta_{jk}}{2\zeta}\{(p_i|A_0|s)^{(0)}-(p_i|A_0|s)^{(1)}\}\)
\item \((d_{ij}|A_0|d_{kl})^{(0)} = (P_l-B_l)(d_{ij}|A_0|p_k)^{(0)} - (P_l-C_l)(d_{ij}|A_0|p_k)^{(1)}\)
\(+\frac{\delta_{il}}{2\zeta}\{(p_j|A_0|p_k)^{(0)} - (p_j|A_0|p_k)^{(1)}\}\)
\(+\frac{\delta_{jl}}{2\zeta}\{(p_i|A_0|p_k)^{(0)} - (p_i|A_0|p_k)^{(1)}\}\)
\(+\frac{\delta_{kl}}{2\zeta}\{(d_{ij}|A_0|s)^{(0)} - (d_{ij}|A_0|s)^{(1)}\}\)
\end{itemize}

The new symbols are

\begin{itemize}
\item \((s||s)\) is the overlap integral between two s type Cartesian Gaussian functions \((s||s) = (\pi/\zeta)^{3/2} exp(-\xi(A-B)^2)\)

\item The Boys function is evaluated on \(U = \zeta(P-C)^2\)
\end{itemize}
\section{STO-3G basis set}
\label{sec:org3efa7ca}

\begin{itemize}
\item Slater-type functions are nice for atom orbitals

\item For molecules it is preferred to use a gaussian-type function to make the integrals easier to evaluate, this provides less precision than Slater-types, but for molecules it doesn't matter

\item The normalized 1s Gaussian-type function, centered at \(R_A\), has the form

\[ \phi_{1s}^{GF} (\alpha, r-R_A) = (2\alpha/\pi)^{3/4} e^{-\alpha|r-R_A|^2} \]

where \(\alpha\) is the gaussian orbital exponent
\end{itemize}

The two-electron integals are of the form

\[ (\mu_A\nu_B|\lambda_C\sigma_D) = \int \dd{r_1}\dd{r_2} \phi_{\nu}^{A * }(r_1)\phi_{\nu}^B(r_1)r_{12}^{-1}\phi_{\lambda}^{C * }(r_2)\phi_{\sigma}^D(r_2) \]

The product of two 1s Gaussian functions, each on different centers, is, apart from a constant, a 1s Gaussian function on a third center

\begin{align*}
    \phi_{1s}^{GF}(\alpha, r-R_A)\phi_{1s}^{GF}(\beta,r-R_B) = \qty(\frac{4\alpha\beta}{\pi^2})^{3/4} \text{exp}(-\alpha|r-R_A|^2 - \beta|r-R_B|^2)
\end{align*}

we can expand the exponent, with \(p=\alpha+\beta\)

\begin{align*}
    -\alpha|r-R_A|^2 - \beta|r-R_B|^2 &= -(\alpha+\beta)|r|^2 + (2\alpha R_A+2\beta R_{\beta})\cdot r - (\alpha|R_A|^2 + \beta|R_B|^2) \\
    &= -p|r|^2 + 2pr \cdot \frac{\alpha R_A+\beta R_B}{\alpha+\beta} - p \Bigg| \frac{\alpha R_A + \beta R_B}{\alpha+\beta} \Bigg|^2 \\
    & \qq{} + p\Bigg| \frac{\alpha R_A + \beta R_B}{\alpha+\beta} \Bigg|^2 - (\alpha |R_A|^2 + \beta |R_B|^2) \\
    & \text{Haciendo }  R_P = (\alpha R_A + \beta R_B)/(\alpha+\beta)  \\
    -\alpha |r-R_A|^2 - \beta|r-R_B|^2 &= (-p|r|^2 + 2pr \cdot R_P - p |R_P|^2)- \frac{\alpha\beta}{\alpha+\beta} (|R_A|^2 + |R_B|^2 - 2R_A \cdot R_B) \\
    &= - p |r-R_P|^2 - \frac{\alpha\beta}{\alpha+\beta}|R_A-R_B|^2
\end{align*}

on the other hand

\begin{align*}
    \qty(\frac{4\alpha\beta}{\pi^2})^2 &= \qty(\frac{2\alpha\beta}{(\alpha+\beta)\pi} \frac{2(\alpha+\beta)}{\pi})^{3/4} \\
    &= \qty(\frac{2\alpha\beta}{(\alpha+\beta)\pi})^{3/4}\qty(\frac{2p}{\pi})^{3/4}
\end{align*}

therefore

\[ \phi_{1s}^{GF}(\alpha,r-R_A)\phi_{1s}^{GF}(\beta,r-R_B)=\qty(\frac{2\alpha\beta}{(\alpha+\beta)\pi})^{3/4} \qty(\frac{2p}{\pi})^{3/4} \text{exp}(-p|r-R_P|^2) \text{exp}\qty(\frac{\alpha\beta}{\alpha+\beta}|R_A-R_B|^2) \]

whith
\[ K_{AB} = \qty(\frac{2\alpha\beta}{(\alpha+\beta)\pi})^{3/4} \text{exp}\qty(-\frac{\alpha\beta}{\alpha+\beta}|R_A-R_B|^2) \]

\[ \phi_{1s}^{GF}(p,r-R_P) = \qty(\frac{2p}{\pi})^{3/4} \text{exp}(-p|r-R_P|^2) \]

we get

\[ \phi_{1s}^{GF}(\alpha,r-R_A)\phi_{1s}^{GF}(\beta,r-R_B)= K_{AB} \phi_{1s}^{GF}(p,r-R_P) \]

As a result, the four-center integral reduces, for 1s Gaussians, to the two-center integral

\[ (\mu_A\nu_B|\lambda_C\sigma_D) = K_{AB}K_{CD} \int \dd{r_1}\dd{r_2} \phi_{1s}^{GF}(p,r_1-R_P)r_{12}^{-1}\phi_{1s}^{GF}(q,r_2-R_Q) \]

which can be readily evaluated

Since Gaussian functions are not optimum basis functions we use as a basis fixed linear combinations of the primitive Gaussian functions \(\phi_p^{GF}\). These linear combinations, called contractions, lead to contracted gaussian functions (CGF)

\[ \phi_{\mu}^{CGF}(r-R_A) = \sum_{p=1}^L d_{p\mu}\phi_p^{GF}(\alpha_{p\mu, r-R_A}) \]

where \(L\) is the length of the contraction and \(d_{p\mu}\) is a contraction coefficient. The pth normalized primitive Gaussian \(\phi_P^{GF}\) in the basis function \(\phi_{\mu}^{CGF}\) has a functional dependence on the Gaussian orbital exponent (contraction exponent) \(\alpha_{p\mu}\). The idea is to choose in advance the contraction length, contraction coefficients, and contraction exponents that fit the CGF to a desirable set of basis functions \(\phi_{\mu}^{CGF}\).

This procedure is commonly applied to fitting a Slater-type orbital (STO) to a linear combination of \(N = 1,2,3,...\) primitive Gaussian functions, the STO-NG procedure. In particular we'll use the STO-3G basis set.

First we consider fitting a Slater function having Slater exponent \(\zeta = 1\). Considering contractions up to length three so that the three fits we seek to find are

\begin{align*}
    \phi_{1s}^{CGF}(\zeta=1.0, STO-1G) &= \phi_{1s}^{GF}(\alpha_{11}) \\
    \phi_{1s}^{CGF}(\zeta=1.0, STO-1G) &= d_{12}\phi_{1s}^{GF}(\alpha_{12}) + d_{22} \phi_{1s}^{GF}(\alpha_{22}) \\
    \phi_{1s}^{CGF}(\zeta=1.0, STO-1G) &= d_{13}\phi_{1s}^{GF}(\alpha_{13}) + d_{23} \phi_{1s}^{GF}(\alpha_{23}) + d_{33}\phi_{1s}^{GF}(\alpha_{33})
\end{align*}

where the \(\phi_1^{CGF} (\zeta = 1.0,STO-NG)\) are the basis functions that approximate as best as possible a Slater-type function with \(\zeta = 1.0\). So we need to find the coefficients \(d_{p\mu}\) and exponents \(\alpha_{p\mu}\) that provide the best fit. The fitting criterion is one that fits the contracted Gaussian function to the Slater function in a least-squares sense, i.e., by minimizing the integral

\[ I = \int \dd{r}[\phi_{1s}^{SF}(\zeta = 1.0,r) - \phi_{1s}^{CGF}(\zeta = 1.0, STO-NG,r)]^2 \]
equivalently, since the two functions are normalized, one can maximize the overlap between the two functions, i.e., one maximizes

\[ S = \int \dd{r}\phi_{1s}^{SF}(\zeta = 1.0,r) \phi_{1s}^{CGF}(\zeta = 1.0, STO-NG,r) \]

For the STO-1G case, there are no contraction coefficientes, and we only need to find the primitive Gaussian exponent \(\alpha\) which maximizes the overlap

\[ S = (\pi)^{-1/2} (2\alpha/\pi)^{3/4} \int \dd{r}e^{-r}e^{-\alpha r^2} \]


DO YOUR OWN ALGORITHM TO GET YOUR OWN TABLE

\begin{center}
\begin{tabular}{rllr}
\hline
\(\alpha\) &  &  & S\\
0.1 &  &  & 0.8641\\
0.2 &  &  & 0.9673\\
0.3 &  &  & 0.9772\\
0.4 &  &  & 0.9606\\
0.5 &  &  & 0.9355\\
\hline
\end{tabular}
\end{center}

The optimum fit occurs for \(\alpha = 0.270950\). 


WHEN I DO MY OPTIMIZATION for the STO-2G and STO-3g cases, I should get

\begin{align*}
    \phi_{1s}^{CGF}(\zeta = 1.0, STO-1G) &= \phi_{1s}^{GF}(0.270950)\\
    \phi_{1s}^{CGF}(\zeta = 1.0, STO-2G) &= 0.678914 \phi_{1s}^{GF}(0.151623) + 0.430129 \phi_{1s}^{GF}(0.851819)\\
    \phi_{1s}^{CGF}(\zeta=1.0, STO-3G) &= 0.444635 \phi_{1s}^{GF}(0.109818) + 0.535328 \phi_{1s}^{GF}(0.405771) + 0.154329\phi_{1s}^{GF}(2.22766)
\end{align*}
\end{document}
